\documentclass[a4paper, 10pt, french]{article}
% Préambule; packages qui peuvent être utiles
   \RequirePackage[T1]{fontenc}        % Ce package pourrit les pdf...
   \RequirePackage{babel,indentfirst}  % Pour les césures correctes,
                                       % et pour indenter au début de chaque paragraphe
   \RequirePackage[utf8]{inputenc}   % Pour pouvoir utiliser directement les accents
                                     % et autres caractères français
   % \RequirePackage{lmodern,tgpagella} % Police de caractères
   \textwidth 17cm \textheight 25cm \oddsidemargin -0.24cm % Définition taille de la page
   \evensidemargin -1.24cm \topskip 0cm \headheight -1.5cm % Définition des marges
   \RequirePackage{latexsym}                  % Symboles
   \RequirePackage{amsmath}                   % Symboles mathématiques
   \RequirePackage{tikz}   % Pour faire des schémas
   \RequirePackage{graphicx} % Pour inclure des images
   \RequirePackage{listings} % pour mettre des listings
% Fin Préambule; package qui peuvent être utiles

\title{Rapport de TP 4MMAOD : Génération d'ABR optimal}
\author{
BROSSY Pierre (SLE)
\\ EL HAIK Léa (ISSC)
}

\begin{document}

\maketitle


%%%%%%%%%%%%%%%%%%%%%%%%%%%%%%%%%%%%%%%%%%%%%%
\section{Principe de notre  programme}

Dans un premier temps, notre programme lit le fichier afin de récupérér les entiers constituant l'arbre. \\
A partir de ces valeurs, on crée un tableau de taille \texttt{n} pour stocker les fréquences de recherche des entiers lus. Les fréquences sont stockées dans le un tableau de type \texttt{(double*)}
Par un principe de memoïsation, on va en déduire le tableau des coûts et le tableau des racines des ABR possibles. Ces tableaux sont de type \texttt{(double **)} sauf le tableau des racines qui est un \texttt{(long **)} car on a besoin d'utiliser les valeurs contenues dans ce tableau comme indices de tableau. Et le langage C n'autorise que les types \texttt{int} ou \texttt{long} en indice de tableau de plus nous avons instancié \texttt{BSTTree} en \texttt{long**} au lieu d'un \texttt{int**} pour pouvoir mieux gérer les grand nombres.\\
Ensuite, en prenant en compte qu'un sous ABR d'un ABR optimal est optimal, on peut utiliser la récursivité afin de trouver l'ABR optimal. \\

Concrétement, nous avons rajouté un package \texttt{BST.c} et \texttt{BST.h} qui nous permettent de créer, manipuler et libérer ces tableaux. Les tableaux utilisés sont :\\
\begin{itemize}
\item \texttt{tab2f} La matrice triangulaire supérieure initialisée dans mémorisation
\item \texttt{tab_cout} La matrice initialisée dans mémorisation qui nous permet de stocker les résultats de la mémoïsation.
\item \texttt{racine} L'arbre calculé dans \texttt{BST\_rec} qui contient les positions des racines de l'arbre. La racine de L'ABR optimal est \texttt{racine[i,j]}
\end{itemize}

Les fonctions principales sont implémentées en récursif et sont les deux fonctions $BST\_rec$ et $BST\_fin$. La première permet d'initialiser le tableau à deux dimensions des racines à l'aide du tableau de la matrice des fréquences et du tableau des coûts. \\

$BST\_fin$ Quand à elle va remplir les valeurs de $BSTTree$ à l'aide de l'arbre $racine$ récursivement.
%%%%%%%%%%%%%%%%%%%%%%%%%%%%%%%%%%%%%%%%%%%%%%
\section{Analyse du coût théorique (2 points)}
{\em Donner ici l'analyse du coût théorique de votre programme en fonction du nombre $n$ d'éléments dans le dictionnaire.
 Pour chaque coût, donner la formule qui le caractérise (en justifiant brièvement pourquoi cette formule correspond à votre programme),
 puis l'ordre du coût en fonction de $n$ en notation $\Theta$ de préférence, sinon $O$.}

  \subsection{Nombre  d'opérations en pire cas\,: }
    \paragraph{Justification\,: }
    {\em La justification peut être par exemple de la forme: \\
       "Le programme itératif contient les boucles $k_1=...$, $k_2= ...$ etc correspondant à la somme
      $$T(n_1, n_2, c_1, c_2) = \sum_{k_1=...}^{...} ... \sum ... + \sum_{i=...}^{...} ...$$
      somme que nous avons calculée (ou majorée) par la technique de  ... " \\
      ou  encore\,:  \\
      "les appels récursifs du programme permettent de modéliser son coût par le système d'équations aux récurrences
      $$T(k_1, k_2) = ...  \mbox{~avec~les~conditions~initiales~....~} $$
      Le coût indiqué est obtenu en résolvant ce système par la méthode de  .... "
    }
  \subsection{Place mémoire requise\,: $\Theta(n^2)$ }
    \paragraph{Justification\,: }
    On stocke les valeurs dans 4 tableaux à deux dimensions. On a également un tableau de dimension $n$ et plusieurs variables annexes
  \subsection{Nombre de défauts de cache sur le modèle CO\,: }
    \paragraph{Justification\,: }
    Soit $Z$ la taille du cache et $L$ la taille d'un bloc, c'est-à-dire que le cache contient $Z/L$ blocs (modèle CO).
Dans la fonction \texttt{memorisation}, la partie triangulaire supérieure de la matrice \texttt{tab2f} est parcourue selon les lignes, ce qui engendre $\frac{n}{L}$ défauts de cache.
Une exécution de la boucle \texttt{for} de la fonction \texttt{BST\_rec} avec comme paramètres $i$ et $j\geq i$ provoque $\frac{j-i}{L}$ défauts de cache, soit au total: $$\sum_{0\leq i\leq j\leq n}\frac{j-i}{L}=\frac{1}{L}\sum_{k=0}^n k=\frac{n(n+1)}{2L}$$ défauts de cache. On en déduit au total $\Theta(n^2)$ défauts de cache.



%%%%%%%%%%%%%%%%%%%%%%%%%%%%%%%%%%%%%%%%%%%%%%
\section{Compte rendu d'expérimentation}
  \subsection{Conditions expérimentaless}


    \subsubsection{Description synthétique de la machine\,:}
Nous avons exécutés nos tests sur une machine avec Xubuntu. Les tests ont été compilés avec gcc 6.2.1.
L'ordinateur tourne avex un processeur intel Pentium et possède 4Gb de RAM.
L'ordinateur avait une connexion Internet désactivée , très peu de processus en train de tourner, en soit il a été monopolisé pour ces tests.
Le temps a été mesuré à l'aide de la commande \texttt{/usr/bin/time} en utilisant le temps réel qui s'est déroulé.

    \subsubsection{Méthode utilisée pour les mesures de temps\,: }
Nous avons mesuré le temps execution à l'aide de la fonction \texttt{/usr/bin/time}.
Nous ne prenions en compte que le temps réél. Nous appelions l'exécutable de la facon suivante.\\
\texttt{time ./bin/compileBST <taille> <benchmark>}\\
	Nous avons testé les benchmarks un par un : 5 itérations sur benchmark 1, puis benchmark2 etc


  \subsection{Mesures expérimentales}

    \begin{figure}[h]
      \begin{center}
        \begin{tabular}{|l||r||r|r|r||}
          \hline
          \hline
            & temps     & temps   & temps \\
            & min       & max     & moyen \\
          \hline
          \hline
            benchmark1      &  0.003s   & 0.009s    & 0.006s    \\
          \hline
            benchmark2       & 0.002s   & 0.049     & 0.012s     \\
          \hline
            benchmark3       & 10.108s  & 10.459s   & 10.180s    \\
          \hline
            benchmark4       & 75.958s     &  82.873s    & 81.405s    \\
          \hline
            benchmark5       & 252.091s    &  284.643s   & 276.825s     \\
          \hline
            benchmark6       & 1243.801s   & 1482.878s   & 1370.06s     \\
          \hline
          \hline
        \end{tabular}
        \caption{Mesures des temps minimum, maximum et moyen de 5 exécutions pour les 6 benchmarks.}
        \label{table-temps}
      \end{center}
    \end{figure}

\subsection{Analyse des résultats expérimentaux}
Comme on peut le voir sur le relevé (disponible dans le fichier rapport). On voit que les résultats des tests se rapprochent d'un coût en temps de $\Theta(n^3)$. On a donc bien la pratique qui se rapproche du modèle théorique



\end{document}
%% Fin mise au format

